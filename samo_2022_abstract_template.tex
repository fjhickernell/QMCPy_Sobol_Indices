%  LaTeX template for abstract submission for SAMO 2022
% 
% First name and name of the speaker.
\speaker{Aleksei}{Sorokin}%
%  (put no space here)
% Title of the talk, capitalized.
\title{Efficiently Computing Sensitivity Indices in QMCPy}

% For each author, give the first name, family name, affiliation, and email.
% Ideally, the affiliation and email should fit on a single line.  
% No need to put the full snail mailing address.  
%  One line per author 
\author{Aleksei}{Sorokin}{Applied Mathematics, Illinois Institute of Technology, USA}{asorokin@hawk.iit.edu}
\author{Jagadeeswaran}{Rathinavel}{Applied Mathematics, Illinois Institute of Technology, USA}{jrathin1@hawk.iit.edu}


% Type your abstract here.
\abstract{

%Functional ANOVA (analysis of variance) decomposes a function $f \in \mathcal{L}^2([0,1]^d)$ into the sum of orthogonal sub-functions dependent on only a subset of inputs: $f(\bx) = \sum_{u \in \{1,\dots,d\}} f_u(\bx_u)$ \cite[Appendix A]{mcbook}. This formulation allows the variance of the function, $\text{var}(f)=\sigma^2$, to be decomposed into the sum of variances of sub-functions: $\sigma^2 = \sum_{u \in \{1,\dots,d\}} \sigma^2_u$ where $\sigma^2_u = \text{var}(f_u)$. Sobol' indices $\underline{\tau}_u^2 = \sum_{v \subset u} \sigma^2_v$ and $\overline{\tau}_u^2 = \sum_{v \cap u \neq \emptyset} \sigma^2_v$ quantify the contributions of subsets of $u$ and subsets containing $u$ respectively. The normalized forms, $\underline{\tau}_u^2/\sigma^2$ and  $\overline{\tau}_u^2/\sigma^2$, called the \emph{closed} and \emph{total} sensitivity indices, quantify the proportion of variance explained by a given subset of inputs. These importance scores have been used in a variety of applications for global sensitivity analysis. This talk details how sensitivity indices can be efficiently approximated using QMCPy \cite{qmcpy}, an open source Quasi-Monte Carlo library in Python.  


Sobol' indices quantify the importance of a function's inputs to explaining the output's variance \cite[Appendix A]{mcbook}.  Normalized Sobol' indices, or sensitivity indices, have been used in a variety of applications for global sensitivity analysis. Monte Carlo methods present an efficient approach for approximating these importance scores. In this talk, we will describe our implementation of such techniques into QMCPy \cite{QMCPy}, an open source Quasi-Monte Carlo library in Python. QMCPy utilizes algorithms from \cite{reliable_sobol_indices_approx} to adaptively select an appropriate number of samples so the approximation is guaranteed to be within an desired tolerance of the true sensitivity indices. 

%  If you have references, put them here in a format like below. 
%  This can be obtained using BiBTeX with the bib style plain.bst, uncommenting first the two next lines and replacing them by the generated .bbl file
%\bibliographystyle{plain} 
%\bibliography{ags.bib}
% 
%  Note that this bibliography must be placed inside the abstract.
\begin{thebibliography}{1}

\bibitem{QMCPy}
Sou-Cheng~T. Choi, Fred~J. Hickernell, R.~Jagadeeswaran, Michael~J. McCourt,
  and Aleksei~G. Sorokin.
\newblock {QMCPy}: A {Q}uasi-{M}onte {C}arlo {P}ython library, 2020+.

\bibitem{reliable_sobol_indices_approx}
Llu{\'i}s~Antoni Jim{\'e}nez~Rugama and Laurent Gilquin.
\newblock {Reliable error estimation for Sobol' indices}.
\newblock {\em {Statistics and Computing}}, 28(4):725--738, July 2018.

\bibitem{mcbook}
Art~B. Owen.
\newblock {\em Monte Carlo theory, methods and examples}.
\newblock 2018.


\end{thebibliography}
}  % End of abstract.
