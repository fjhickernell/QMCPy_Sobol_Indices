\documentclass{article}[12pt]
\usepackage[a4paper, margin=1in]{geometry}

\usepackage{url}
\usepackage{hyperref}
\hypersetup{
    colorlinks=true,
    linkcolor=blue,
    filecolor=magenta,      
    urlcolor=blue}
\usepackage{xcolor}
\usepackage{float}
\usepackage{amsmath}
\usepackage{amssymb}
\usepackage{algpseudocode}
\usepackage{algorithm}
\usepackage{graphicx}
\usepackage{booktabs}
\usepackage{amsthm}

\newcommand{\RefereeTODO}[1]{{\color{red} #1 \newline}}
\newcommand{\Referee}[1]{{\color{blue} #1 \newline}}
\newcommand{\MISCComment}[1]{{\color{purple} #1}}

\title{Response to Referee}
\author{Aleksei G. Sorokin, J. Rathinavel}
\date{}

\begin{document}

\maketitle    

\MISCComment{
    \section*{TODO for Authors}
    \begin{itemize}
        \item switch CMC to MC and delineate between MC and QMC throughout
        \item Update bibliography bibitems to include \texttt{hoeffding1992class}, \texttt{snoek2012practical}, \texttt{rasmussen2006gaussian}
        \item Increase font size of figures. 
        \item Typeset references 
        \item Get rid of all warnings 
        \item Reread all for consistency. 
    \end{itemize}}

We thank the referee for their valuable  feedback. We have incorporated all suggestions into the updated draft. 

\section*{Abstract}

\begin{itemize}
    \item \Referee{I would not say "Monte Carlo methods present an *efficient* approach for approximating the expected value of a random variable."; maybe replace "efficient" by "convenient"?}Replaced.
    \item \Referee{``... vector functions of multiple expectations'': this seems to indicate twice the same; or is there still a function applied to the vector of expectations afterwards? Actually it is: I suggest the authors make more clear at the beginning of the abstract what they mean with ``vector functions of integrals''.}The sentence is reworded to now end with ``supports adaptive sampling to satisfy error criteria for functions of a common vector of expectations''.
    \item \Referee{I suggest the authors also refer explicitly to QMC in the abstract, maybe be saying that when they say Monte Carlo that this includes QMC?}The article  now uses Monte Carlo in the classic sense of IID Monte Carlo (what we called Crude Monte Carlo). Quasi-Monte Carlo no longer falls under the umbrella of Monte Carlo.  We now mention both MC and QMC in the first sentence of the abstract. 
    \item \Referee{Maybe the title is not the best possible title for this article; it might be better to explicitly refer to how error estimators for the individual approximations are propagated to the error estimators for the vector valued function of expectations.}The title is changed to better reflect a focus on error prorogation and explicitly mention Quasi-Monte Carlo. 
\end{itemize}

\section*{Section 1}

\begin{itemize}
    \item \Referee{I would not call the function $f$ a ``simulation'', see also later. This word is used throughout the article.}Changed ``simulation'' to ``integrand''.
    \item \Referee{It is not clear if the assumption is that $\vec{f}$ can really be interpreted as $\vec{d}_{\vec{\mu}}$ more or less ``independent'' functions, or not. E.g. in Algorithm 1 it is assumed that I can choose to only calculate certain output elements of $\vec{f}$, but maybe $\vec{f}$ is the discretized solution to a PDE and so, even if I only need the value at one node, I will have to calculate the values at all nodes. I am assuming both of these situations are fine, but the authors might want to clarify this.}The second paragraph in Section 6 now clarifies that economical function evaluation is only a feature for integrands which do not need to calculate values at all nodes. 
    \item \Referee{"... is described *in* Section 6 ..."}Fixed. 
\end{itemize}

\section*{Section 2}

\begin{itemize}
    \item \Referee{``LD sequences implemented in base 2'': remove the ``implemented''.}Removed. 
\end{itemize}

\section*{Section 3}

\begin{itemize}
    \item \Referee{Since the discussion in the paper only discusses first order convergence for QMC the remark that lattice schemes are worse since they require ``periodicity'' is incorrect. Randomly shifted lattice rules give provable order 1 for the function assumed in this manuscript, without any periodization. Using tent-transformed lattices even gives order 2 if the function is sufficiently smooth (without requiring ``periodicity''). So I suggest the authors change or update the last paragraph in this section.}This paragraph has been removed.
    \item \Referee{Reference [26] seems to be an unpublished reference.}This was a reference to \url{https://artowen.su.domains/mc/} with the suggested Bibex entry. We removed the reference and replaced it with citations from within the book.  
\end{itemize}

\section*{Section 4}

\RefereeTODO{I think I do not understand this section, so the authors will need to rewrite
this section. They can be helped by some of my questions}

\begin{itemize}
    \item \RefereeTODO{Please rewrite / reorder the beginning of section 4. Start with restating that you are computing $s = C(\vec{\mu})$, then state the desired $\alpha^{(s)}$, and then state how for every multi-index $\vec{k}$ you can have bounds as in (2). Currently the order is very confusing.}
    \item \RefereeTODO{Maybe this section can be formalized into a lemma.}
    \item \RefereeTODO{For equations (2) and (3): what is the benefit of writing this for any multi-index $\vec{k}$? Why not just write this for $\vec{\mu}$?}
    \item \RefereeTODO{Also: in the rhs of (3) there is no $\vec{k}$ and this evaluates to a scalar; so I guess there is something wrong with this formula as the authors say "individual uncertainty levels" while they are the same for every $\vec{k}$.}
    \item \RefereeTODO{This is my understanding of $\vec{\mu} \in \mathbb{R}^{\vec{d}_{\vec{\mu}}}$: this is just the "reshape" into a multi-dimensional array of $L = \prod(\vec{d}_{\vec{\mu}})$ expectations. So e.g. it could be that we have 12 expectations and that they are arranged into an array of 2x6 or maybe 2x2x3.}
    \item \RefereeTODO{Then when the authors write $\mu_{\vec{k}}$ for some multi-index $\vec{k}$ I interpret this as the corresponding index of the expectation in that multi-dimensional array.}
    \item \RefereeTODO{In this way I interpret (2) as making a statement about a particular expectation, the one with index $\vec{k}$, to be in a certain confidence interval with probability $1 - \alpha_{\vec{k}}^{(\vec{\mu})}$.}
    \item \RefereeTODO{All these probabilities are gathered together again in a multi-dimensional array $\vec{\alpha}^{(\vec{\mu})}$. And according to my interpretation of (3) this is a constant array.}
    \item \RefereeTODO{Equation (4) then states what the probability is of all the individual confidence bounds to hold all at the same time. This is done through saying that the probability of the confidence interval failing for expectation $\vec{k}$ is given by $\alpha_{\vec{k}}^{(\vec{\mu})}$, so the probability of all of them failing is the sum which is
    $$\sum_{\vec{k}} \alpha_{\vec{k}}^{(\vec{\mu})} = \alpha^{(s)}$$
    So the authors just force all of the individual probabilities to be $L$ times smaller, such that adding all L failure probabilities is equal to $\alpha^{(s)}$. This is equation (4).}
    \item \RefereeTODO{At this point I think this could have all been said in one or two sentences. E.g.: To guarantee ... equation (4) ... it is sufficient to make the probability of not being in the correct confidence interval for each of the individual expectations $L$ times smaller where $L$ is the total number of expectations to estimate. I do not understand why the notation has to be so complicated.}
    \item \RefereeTODO{In the last paragraph of this section: ``then use vectorized, scalar MC algorithms to find individual bounds $[\vec{\mu}^-, \vec{\mu}^+]$ satisfying (2) ...'': I do not understand the ``vectorized, scalar'', and I think the authors meant to write ``to find approximations for the different $\mu_{\vec{k}}$'s such that ...???...''; what is meant with the ``individual bounds'' and ``$[\vec{\mu}^-, \vec{\mu}^+]$''?}
    \item \RefereeTODO{At the end of this section I start to see part of my confusion rests in the fact that the authors write $\vec{\mu}$ here as the true value, but in $[\vec{\mu}^-, \vec{\mu}^+]$ is where the calculated values are. Maybe the word usage of ``solution'' in the beginning of section 4 is not such a good choice as it suggested to me that it my confusing as to whether the word "solution" is used for the ``approximation'' or for the ``true value''.} 
    \item \RefereeTODO{In light of what happens in Section 6 later: I suggest the authors make a graphical illustration as to how values from $\vec{\mu}$ end up in combined values in $\vec{s}$ and as to how the dependency function $\vec{D}$ captures this. In that same graphical illustration it should be clear which are the ``true values'' and how the algorithms obtain ``intervals'' which then get combined into computed intervals for the combined approximations. I think it is best to not use the word "solution" anywhere in this context.}
\end{itemize}

\section*{Section 5}

\begin{itemize}
    \item \Referee{For equation (6) I think it is better to write this as
    $$\sup_{s \in [s^-, s^+]} |s - \hat{s}| \le h^{(\epsilon)}(s)$$
  Otherwise inattentive readers might think $s$ is the true solution.}Changed to the suggested equation. 
  \item \Referee{``i.e '' $\to$ ``i.e.''}Fixed.
\end{itemize}

\section*{Section 6}

\subsection*{Algorithm 1}

\begin{itemize}
    \item \Referee{The word ``simulation'' for $f$ might be confusing.}Changed to ``integrand'' throughout.
    \item \Referee{To me it is not clear if ``Require'' is the same as ``Input'' or if it is a pre-condition for calling this algorithm that needs to be satisfied. Clearly $f$ is an input, and I think all the ``require'' lines refer to inputs to be specified by the user. The confusing part comes from the fact that also the post-conditions of the algorithm are specified. E.g. ``... satisfy $P(\vec{s}_{\vec{l}} \in [s_{\vec{l}}^-, s_{\vec{l}}^+]) \ge 1 -\alpha_{\vec{l}}^{(\vec{s})}$''.}Changed \textbf{Require} to \textbf{Input} and changed description of  input $\boldsymbol{\alpha}^{(\boldsymbol{s})} \in (0,1)^{\boldsymbol{d}_{\boldsymbol{s}}}$ to clarify the inputs effect on the returned values of the algorithm.
    \item \Referee{Am I required to provide $\vec{C}^-$ and $\vec{C}^+$ or is this assumed to be done by some interval arithmetic package? Or is this not possible since $\hat{s}_{\vec{l}}$ is not just $C_l([\vec{\mu}^-,\vec{\mu}^+])$?}Currently, these algorithm inputs must be specified by the user. While $s_{\boldsymbol{l}} \neq C_{\boldsymbol{l}}([\boldsymbol{\mu}^-,\boldsymbol{\mu}^+])$ generally, $\boldsymbol{C}^-$, $\boldsymbol{C}^+$, and $\boldsymbol{D}$ could feasibly be determined from $\boldsymbol{C}$, but we have not done so here. We have updated the last section (renamed to ``Discussion and Further Work'') to mention future work in automatically determining $\boldsymbol{C}^-$, $\boldsymbol{C}^+$, and $\boldsymbol{D}$ from $\boldsymbol{C}$.
    \item \Referee{In the comments: write ``stopping flags'' instead of ``flags'' (2 times).}Done.
    \item \Referee{The algorithm says to set ${\vec{\alpha}}^{(\vec{\mu})}$ ``as described in Section 6''; this is not described in section 6?}The  description in Section 6 is improved as suggested in the ``Other Section 6'' comment below. 
    \item \Referee{At the end of the algorithm the authors return $\vec{s}^-$ as $\hat{\vec{s}}^-$, the same remark for $\vec{s}^+$. I guess this is from previous notation in which the $\hat{}$'s denote ``computed'' values from the integral approximations.}This was a typo, removed the hats.
\end{itemize}

\subsection*{Other Section 6}\label{sec:os6}

\begin{itemize}
    \item \Referee{``Moreover,*space missing* $\vec{D}$ may be used ...'': space missing; and this is not very clear, I would have expected this to be explained in Section 4, but I am confused in Section 4. I see that this might be were the authors are referring to from the algorithm, but also that is not very clear. How will I do this? Please give formulas if possible. It should at least be more prominently present if you are referring to this from the algorithm.}Added extra space. Section 4 only concerns a scalar combined solution, so a dependency function is irrelevant. The dependency function only becomes relevant when you want to tell the integrand to avoid computing certain outputs as they only contribute to a combined solution that is already approximated sufficiently well. We have updated the description in Section 6 to give a clearer idea of how to set individual uncertainties and a bit of technical detail too. 
\end{itemize}

\section*{Section 7}

\begin{itemize}
    \item \Referee{Reference [12] seems to be just an arxiv preprint, there must be a better source to cite.}Unfortunately, this is the only reference to this helpful article. We've replaced with other non-arxiv references.
    \item \Referee{``are the rows of argmax ...'': argmax will just return one element; here it is confusing as you do argmax over a square domain, but even then it should just return one row (or one column depending on how you define this). Argmax implies some kind of looping, but $\vec{Z}$ is not defined (since it is an element of this square domain). It would help if you first say what $\alpha$ is, namely it is a function which considers q vectors $\vec{z}$ at the same time; but you could resolve this easily by writing $\alpha(\vec{w}_1, \ldots,\vec{w}_q)$ and then $\mathrm{argmax}_{\vec{w}_1,\ldots,\vec{w}_q \in [0, 1]^d}$ $\alpha(\vec{w}_1, \ldots, \vec{w}_q)$.}We have reset the notation to be closer aligned with the rest of the text. $d$ is the number of next points to sample (previously $q$), and $d_\mu$ (previously $c$) is the number of candidates, each a matrix of size $d \times \nu$ where $\nu$ is dimension of the domain over which the black box function is to be optimized. The one element argmax returns is a matrix. The suggested notation would work for the equation in question but would be clunky when considering multiple candidate matrices as done for optimizing over a discrete set of matrices. 
    \item \Referee{The vectors $\vec{z}_1, ...$ in the previous paragraph are over $[0, 1]^\nu$, here they seem to be from $[0, 1]^d$; please write explicitly and if necessary correct.}Addressed in previous response. 
    \item \Referee{``posterior mean *and* covariance'' (instead of a comma).}Updated in two places. 
    \item \Referee{``at points $\vec{Z}$'': I would suggest writing this as ``at points $\vec{w}_1,\ldots,\vec{w}_q$ given the data at the points $\vec{z}_1,\ldots,\vec{z}_N$''.}Addressed in previous response.
    \item \Referee{The authors suddenly go one-dimensional when explaining QEI.}The example is one dimensional in the domain of the black-box function so we are able to plot the idea.
    \item \Referee{Should the ``q'' in ``q-Expected'' be in math mode?}No, qEI is a common name for the acquisition function. Plus, we have changed what was previously $q$ into $d$ for consistency with the rest of the paper. 
    \item \Referee{From the next paragraph on you can introduce the notation $\vec{Z}$ to group the new points.}We have opted to introduce the notation earlier for easier comparison between the two argmax expressions: the one over the continuous domain and the one over the discrete domain. Let us know if this is still confusing and we can update accordingly. 
    \item \Referee{Am I understanding correctly that $c$ in the example is like $32^2$?}Yes, $c$ (which is now $d_\mu$) is the number of white dots in the right plot of Figure 2, so $32^2$.
    \item \Referee{Write $\vec{\theta} \in \vec{\Theta}$ in the first line of Section 7.2?}$\boldsymbol{\Theta}$ is the random vector of coefficients while we reserve $\boldsymbol{\theta}$ for realizations of $\boldsymbol{\Theta}$. 
    \item \Referee{Do the $\vec{S}$ after the Bayes formula need to be $\vec{\theta}$?}Yes, changed. 
    \item \Referee{Specify the dimensionality of $\vec{\Theta} \subseteq \mathbb{R}^{d_s}$?}Now specified in the first sentence along with the fact that $\boldsymbol{\Theta}$ is the random variable.  
    \item \Referee{``posterior mean of $\vec{\theta}$''?}$\boldsymbol{\Theta}$ is the random variable whose posterior mean we want to compute. 
    \item \Referee{I started reading this section under the impression that $\vec{\Theta}$ was the domain of $\vec{\theta}$ and that there is some distribution on $\vec{\Theta}$. But there seems to be confusion with the authors whether they wanted to write $\vec{\theta}$ or $\vec{\Theta}$.}See previous comments. 
    \item \Referee{Why do the authors opt to calculate $\mu_{1,k}$ explicitly for each $k$ while they are the same for all $k$? I do not understand the ``removes economic function evaluation in favor of reduced storage''. What is the bottle neck? Please explain.}Elaborated explanation in the last paragraph of Section 7.2. 
    \item \Referee{Is [26] a published reference?}No, see previous comment. It has been replaced with a reference within [26]. 
    \item \Referee{``The closed, total Sobol' indices'': the comma reads strange, maybe replace by ``and''?}Replaced commas with ``and... respectively'' throughout the section. 
    \item \Referee{Explain the notation $(\vec{\varphi})$.}Same as going from $f$ to $\boldsymbol{f}$. Mentioned that $\boldsymbol{\varphi}$ may have multi-dimensional output when first introduced. 
    \item \Referee{Figure 3: please increase font size.}Increased font size on all figures.
\end{itemize}

\section*{Section 8}

\begin{itemize}
    \item \Referee{The authors suddenly write the more common ``MC'' instead of ``CMC''.}We have changed to the common MC for IID Monte Carlo and QMC for Quasi-Monte Carlo throughout. 
\end{itemize}

\section*{References}

\begin{itemize}
    \item \Referee{Please check the typesetting of the references.}Done. 
\end{itemize}

\end{document}
