\documentclass[]{elsarticle}
\setlength{\marginparwidth}{0.5in}
\usepackage{amsmath,amssymb,amsthm,url,natbib,mathtools,bbm,extraipa,accents,graphicx}
\usepackage{hyperref}

\newcommand{\fudge}{\fC}
\newcommand{\dtf}{\textit{\doubletilde{f}}}
\newtheorem{lem}{Lemma}
\theoremstyle{definition}
\newtheorem{defin}{Definition}
\newtheorem{algo}{Algorithm}
\newcommand{\cube}{[0,1)^d}
\DeclareMathOperator{\trail}{trail}
\newcommand{\rf}{\mathring{f}}
\newcommand{\rnu}{\mathring{\nu}}


\begin{document}

\begin{frontmatter}

\title{Automatic Estimation of Sobol' Indices Based on Quasi-Monte Carlo Methods}
\author{Llu\'is Antoni Jim\'enez Rugama\\
Illinois Institute of Technology}
%\address{Room E1-208, Department of Applied Mathematics, Illinois Institute of Technology,\\ 10 W.\ 32$^{\text{nd}}$ St., Chicago, IL 60616}
\begin{abstract}
In this talk we will introduce a new estimation for the Sobol' indices based on quasi-Monte Carlo methods. Sobol' indices are a variance based sensitivity analysis developed by Ilya M. Sobol'. Given a random variable Y as a function of another set of random variables, these indices measure what part of the variance of Y is explained by each set of input variables. In addition, we will discuss the origin of the quasi-Monte Carlo integration error, and propose a new automatic algorithm that estimates the indices. The key point of the algorithm is that it will find the number of integration points itself, and return the estimate with an error no greater than the user specified tolerance.

\end{abstract}

%\begin{keyword}
%%% keywords here, in the form: keyword \sep keyword
%%Multidimensional integration; Automatic algorithms; Guaranteed algorithms; Quasi-Monte Carlo; Rank-1 lattices; Fast transforms
%%% MSC codes here, in the form: \MSC code \sep code
%%% or \MSC[2008] code \sep code (2000 is the default)
%
%\end{keyword}

\end{frontmatter}

\bibliographystyle{alpha}
%\bibliographystyle{model1b-num-names.bst}
\bibliography{bib}
\end{document}
