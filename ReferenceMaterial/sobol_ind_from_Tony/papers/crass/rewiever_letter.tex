\documentclass[10pt,a4paper]{article}
\usepackage[utf8]{inputenc}
\usepackage[english]{babel}
\usepackage{amsmath}
\usepackage{amsfonts}
\usepackage{amssymb}
\usepackage{color}
\usepackage{graphicx}
\usepackage{dashrule}

\def\dashfill{\cleaders\hbox to 2em{-}\hfill}

\begin{document}

\begin{flushleft}

\textcolor{blue}{Dear Reviewer,\\
we would like to thank you for the positive and constructive review feedback. Below we provide the response (blue text) to the issue raised in the report.\\
Best regards,\\
Laurent Gilquin and co-authors}

\section*{Report on ``Iterative construction of replicated designs based on Sobol' sequences''\\ \small{by Gilquin et al.}}

\textbf{Reviewer}\\\

In this paper replicated experimental designs based on Sobol' sequences are constructed iteratively. Two versions, which allow for iterative design sets that grow linearly or geometrically, are proposed. I find that this is a valuable contribution to the field of experimental design, in particular for the estimation of Sobol indices used in sensitivity analysis.\\\

I have, however, one remark: It is known that a LH sample does not have good space-filling properties. That is why one usually iteratively generates latin hypercube samples to find the best one according to some criterion, typically maximin. This is what standard toolboxes propose by default. Is it the way the LH in section 4 was generated ? If yes, please say it. If not, it should be fair to compare the new designs with such a standard "optimized" LHS.\\\

\textcolor{blue}{The initial LH discussed in the article was not optimized. Thus, we optimized the LH designs according to each criterion ($L^2$ discrepancy and maximin distance) and updated the results for both figures. In addition, we also added a few comments that appear in blue in the article.}

\end{flushleft}

\end{document}